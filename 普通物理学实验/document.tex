\documentclass{ctexart}
\usepackage{geometry}
\usepackage{enumitem}
\usepackage{graphicx}
\usepackage{listings} % 引入宏包
\usepackage{xcolor}   % 引入颜色宏包,用于代码块着色
\usepackage{titlesec}  % 用于自定义标题格式
\usepackage{float}
\usepackage{fancyhdr}%页眉页脚
\usepackage{amsmath}
\usepackage{siunitx}

\geometry{a4paper, left=3cm, right=3cm, top=3cm, bottom=3cm}
% 自定义一级标题的格式,左对齐,中文数字编号
\titleformat{\section}
{\normalfont\Large\bfseries}  % 标题的字体、大小和粗体
{\chinese{section}、}         % 自动生成中文编号(如一、二、三)
{0em}                        % 编号与标题文字之间的间距
{}                           % 标题前缀

% 让 figure 的编号跟随章节编号
\numberwithin{figure}{section}
\numberwithin{table}{section}
\pagestyle{fancy}	

\begin{document}
	
	\lstset{                    % 设置代码显示风格
		language=Verilog,        % 代码语言
		basicstyle=\ttfamily,   % 基本字体设置
		keywordstyle=\color{blue},  % 关键字颜色
		commentstyle=\color{green}, % 注释颜色
		stringstyle=\color{red},    % 字符串颜色
		showstringspaces=false, % 不显示空格符
		numbers=left,                          	% 在左侧显示行号
		frame=single,                         	% 设置代码块边框
	}
	
	
	\thispagestyle{empty}
	\begin{center}
		\includegraphics[width=0.6\textwidth]{logo.png}
	\end{center}
	\vspace{1.8cm}
	
	\graphicspath{{figures/}}
	\begin{center}
		\zihao{-3}
		\begin{tabular}{rl}
			\textbf{课题题目} & : \underline{\makebox[8cm]{xxxxxxx}} \\
			& \; \underline{\makebox[8cm]{(xxxxxxx)}} \\
			\vspace{0.1cm} & \\
			\textbf{姓名学号} & : \underline{\makebox[8cm]{xx\quad xxx}} \\
			\vspace{0.1cm} & \\
			&\;  \underline{\makebox[8cm]{xx\quad xxx}}\\
			\vspace{0.1cm} & \\
			&\;  \underline{\makebox[8cm]{xx\quad xxx}} \\
			\vspace{0.1cm} & \\
			&\;  \underline{\makebox[8cm]{xx \quad xxx}} \\
			\vspace{0.1cm} & \\
			\textbf{指导老师} & : \underline{\makebox[8cm]{xxx}} \\
		\end{tabular}
	\end{center}
	
	\vspace{0.6cm}
	
	\begin{center}
		\zihao{-3}
		\textbf{物理实验教学中心}
	\end{center}
	\begin{center}
		\zihao{-3}
		\textbf{2024 年 12 月 25 日}
	\end{center}
	
	\tableofcontents
	\newpage
	\pagenumbering{arabic}
	
	\section{背景知识}
	
	\newpage
	\section{原理阐述}
	
	
	\newpage
	\section{实验装置}
	
	\subsection{硬件装置}
	
	\subsection{软件装置}
	
	
	\newpage
	\section{实验操作与结果}
	
	\subsection{实验方法}
	
	\subsection{实验数据与结果}
	
	\newpage
	\section{分析与讨论}
	
	\subsection{误差分析}
	
	\subsection{实验分析}
	
	\newpage
	\section{实验总结}
	
	\subsection{实验总述}
	
	\subsection{不足之处}
	
	\subsection{未来展望}
	
	\newpage
	\section{课题总结}
	
	\subsection{仪器系统照片}
	
	\subsection{实验过程记录}
	
	\subsection{个人小结与感想}
	
	\newpage
	\section{参考文献}
	
	% 按照以下格式列出所有参考文献:
	% [1] Author, “Title of the article”, Name of the journal, page no., (year)
	% [2] Author, “Title of the book”, Name of the publisher, year
	% [3] Author, “Title of the article”, Web site links
	
\end{document}