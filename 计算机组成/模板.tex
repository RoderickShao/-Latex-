\documentclass{ctexart}
\usepackage{geometry}
\usepackage{enumitem}
\usepackage{graphicx}
\usepackage{listings} % 引入宏包
\usepackage{xcolor}   % 引入颜色宏包,用于代码块着色
\usepackage{titlesec}  % 用于自定义标题格式

\geometry{a4paper, left=3cm, right=3cm, top=3cm, bottom=3cm}
% 自定义一级标题的格式,左对齐,中文数字编号
\titleformat{\section}
{\normalfont\Large\bfseries}  % 标题的字体、大小和粗体
{\chinese{section}、}         % 自动生成中文编号(如一、二、三)
{0em}                        % 编号与标题文字之间的间距
{}                           % 标题前缀


\begin{document}
	\lstset{                    % 设置代码显示风格
		language=Verilog,        % 代码语言
		basicstyle=\ttfamily,   % 基本字体设置
		keywordstyle=\color{blue},  % 关键字颜色
		commentstyle=\color{green}, % 注释颜色
		stringstyle=\color{red},    % 字符串颜色
		showstringspaces=false, % 不显示空格符
		numbers=left,                          	% 在左侧显示行号
		frame=single,                         	% 设置代码块边框
	}
	
	\pagestyle{empty}
	\begin{center}
		\includegraphics[width=0.6\textwidth]{image.png}
	\end{center}
	\vspace{1.8cm}
	
	\begin{center}
		\zihao{2} \bf{本科实验报告}
	\end{center}
	
	\vspace{2.2cm}
	
	\begin{center}
		\zihao{-3}
		\begin{tabular}{rl}
			课程名称 & : \underline{\makebox[8cm]{计算机组成}} \\
			\vspace{0.15cm} & \\
			姓\hspace{2em}名 & : \underline{\makebox[8cm]{xxx}} \\
			\vspace{0.15cm} & \\
			学\hspace{2em}院 & : \underline{\makebox[8cm]{计算机学院}} \\
			\vspace{0.15cm} & \\
			系 & : \underline{\makebox[8cm]{计算机科学与技术系}} \\
			\vspace{0.15cm} & \\
			专\hspace{2em}业 & : \underline{\makebox[8cm]{计算机科学与技术}} \\
			\vspace{0.15cm} & \\
			学\hspace{2em}号 & : \underline{\makebox[8cm]{xxxxxxx}} \\
			\vspace{0.15cm} & \\
			指导教师 & : \underline{\makebox[8cm]{xxxxx}}
		\end{tabular}
	\end{center}
	
	\vspace{1.8cm}
	\begin{center}
		\zihao{-3}
		2024 年 10 月 21 日
	\end{center}
	\vspace{1cm}
	
	
	\newpage
	\begin{center}
		\zihao{3} \bf 浙江大学实验报告
	\end{center}
	\vspace{1cm}
	\zihao{-4}
	\noindent 课程名称: \underline{\makebox[5cm]{计算机组成}} 
	实验类型: \underline{\makebox[5cm]{综合}} \\[0.4cm] 
	实验项目名称: \underline{\makebox[11.3cm]{实验项目名称}} \\[0.4cm] 
	学生姓名: \underline{\makebox[2cm]{xxx}} 
	专业: \underline{\makebox[4.6cm]{计算机科学与技术}} 
	学号: \underline{\makebox[3cm]{xxxx}} \\[0.4cm]
	同组学生姓名: \underline{\makebox[11.3cm]{xxx、xxx、xxx}}\\[0.4cm]
	指导老师: \underline{\makebox[2.5cm]{xx}} 
	助教: \underline{\makebox[2.5cm]{xxx}} \\[0.4cm]
	实验地点: \underline{\makebox[5cm]{东4-509}} 
	实验日期: \underline{\makebox[1.4cm]{2024}} 年 \underline{\makebox[0.9cm]{10}} 月 \underline{\makebox[0.9cm]{21}} 日
	
	\vspace{1.5cm}
	\graphicspath{{figures/}}
	
	\section{实验目的及环境}
	
	\subsection{实验目的}
	\subsection{实验环境}
	\section{实验目标和原理}
	
	\section{实验内容}
	
	\section{实验结果与分析}
	
	\section{讨论、心得}
	
	\enddocument