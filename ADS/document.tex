\documentclass[a4paper,12pt]{article}
\usepackage{graphicx}
\usepackage{amsmath}
\usepackage{hyperref}
\usepackage{float}
\usepackage{listings}
\usepackage{xcolor}
\usepackage{geometry}
\usepackage{indentfirst}
\usepackage{algorithm}
\usepackage{algpseudocode}
\geometry{left=3cm,right=3cm,top=2.5cm,bottom=2.5cm}

%目的是让伪代码可以分页
\usepackage{lipsum}

\makeatletter
\newenvironment{breakablealgorithm}
{% \begin{breakablealgorithm}
		\begin{center}
			\refstepcounter{algorithm}% New algorithm
			\hrule height.8pt depth0pt \kern2pt% \@fs@pre for \@fs@ruled
			\renewcommand{\caption}[2][\relax]{% Make a new \caption
				{\raggedright\textbf{\ALG@name~\thealgorithm} ##2\par}%
				\ifx\relax##1\relax % #1 is \relax
				\addcontentsline{loa}{algorithm}{\protect\numberline{\thealgorithm}##2}%
				\else % #1 is not \relax
				\addcontentsline{loa}{algorithm}{\protect\numberline{\thealgorithm}##1}%
				\fi
				\kern2pt\hrule\kern2pt
			}
		}{% \end{breakablealgorithm}
		\kern2pt\hrule\relax% \@fs@post for \@fs@ruled
	\end{center}
}
\makeatother

\hypersetup{
	colorlinks=true,
	linkcolor=black
}

%设置图片路径
\graphicspath{{img/}}

% 设置代码样式
\lstset{
	language=C,
	basicstyle=\ttfamily\footnotesize,
	keywordstyle=\color{blue},
	commentstyle=\color{gray},
	stringstyle=\color{red},
	showstringspaces=false,
	numbers=left,
	numberstyle=\tiny\color{gray},
	stepnumber=1,
	numbersep=8pt,
	backgroundcolor=\color{white},
	tabsize=2,
	breaklines=true,
	breakatwhitespace=false
}

% 定义一个新的命令来创建标题页
\newcommand{\makecustomtitle}{
	\begin{titlepage}
		\centering
		\includegraphics[width=0.5\textwidth]{./image.png}\\[2cm]
		{\fontsize{36}{10}\bfseries Advanced Data Structures and Algorithm Analysis}\\[2cm]
		{\Huge\bfseries\linespread{1.5} Project x: xxx}\\[2cm]
		{\large\bfseries\linespread{1.5}\selectfont Date: \today}\\[2cm]
	\end{titlepage}
}

\begin{document}
	
	\makecustomtitle
	
	\tableofcontents
	\newpage
	
	\section{Introduction}
	
	\newpage
	\section{Algorithm Specification}
	
	\begin{algorithm}[H]
		\caption{Example}
		\begin{algorithmic}[1]
			\Function{Start}{}\\
			Print Hello world!
			\EndFunction
		\end{algorithmic}
	\end{algorithm}
	
	
	\newpage
	\section{Testing Results}
	% 测试案例表。每个测试案例通常包括本案例目的的简要描述、预期结果、程序的实际行为、如果程序未按预期工作则可能的错误原因,以及当前状态(“通过”,或“已修正”,或“待定”)。
	
	\newpage
	\section{Analysis and Comments}
	
	\newpage
	\appendix
	\section*{Appendix: Source Code}
	% \lstinputlisting{xxx.c}
	
	\section*{References}
	% 按照以下格式列出所有参考文献:
	% [1] Author, “Title of the article”, Name of the journal, page no., (year)
	% [2] Author, “Title of the book”, Name of the publisher, year
	% [3] Author, “Title of the article”, Web site links
	
	\section*{Author List}
	
	\section*{Declaration}
	% 声明该项目的所有工作都是作为团队独立完成的。
	We hereby declare that all the work done in this project titled "xxx" is of our independent effort as a group.
	
\end{document}